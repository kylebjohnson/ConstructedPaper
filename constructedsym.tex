\documentclass[twoside, 12pt]{article}
\usepackage[normalem]{ulem}
\usepackage[]{fancyhdr}
\usepackage{varioref, amsmath, stmaryrd}
\usepackage[oldstyle]{clara}
\usepackage[T1]{fontenc}
\usepackage[backend=bibtex,natbib,style=authoryear,sorting=nyt]{biblatex}
\addbibresource{Bibliography.bib}
\usepackage{xcolor}
  \colorlet{arrow}{gray}
 \usepackage[colorlinks, urlcolor=blue, linkcolor=black,
  citecolor=black, filecolor=black, pagecolor=black]{hyperref}
\usepackage[norule]{footmisc}
\usepackage{microtype}
\usepackage{tikz}
\usetikzlibrary{decorations.markings}
\usetikzlibrary{arrows}
\usepackage[linguistics]{forest}
\usepackage[all]{nowidow}
\usepackage{gb4e}

% Better looking denotation brackets from stmaryrd
\newcommand{\denotes}[2][]{\ensuremath{\llbracket \text{#2}\rrbracket^{#1}}}

\forestset{
default preamble={for tree={s sep=5mm, inner sep=0, l=0}
}}


\newcommand{\paper}{Constructing Symmetric Predicates}


\title{\paper\thanks{Acknowledgements}}
\author{Jeroen van Craenenbroeck and Kyle Johnson}
\date{\today}

% \widowpenalty=10000
% \clubpenalty=10000

\begin{document}
\begin{titlepage} 
\maketitle 
\thispagestyle{empty}
\begin{abstract}

 \noindent 
 
 \end{abstract}
 \end{titlepage} 

\thispagestyle{empty}
\noindent
There are predicates in many languages that can express a relation between individuals either by taking one plural argument referring to those individuals or two, possibly singular, arguments that together refer to those individuals. This is illustrated in (\ref{ex:external}) and~(\ref{ex:internal}).
\begin{exe}
\raggedright
\ex \label{ex:plainsymm} $aP$ $\Leftrightarrow$ $bPc$, when $a = b \oplus\ c$
\begin{xlist}
  \ex \label{ex:external} intransitive/transitive alternation
  \begin{xlist}
    \ex Arjun and Aisha dated $\Leftrightarrow$ Aisha dated Arjun
    \ex Our names rhyme $\Leftrightarrow$ Your name rhymes with mine
    \ex California and Oregon are adjacent $\Leftrightarrow$ California is adjacent to Oregon
  \end{xlist}
  \ex \label{ex:internal} transitive/ditransitive alternation
  \begin{xlist}
    \ex She mixed camphor and menthol $\Leftrightarrow$ She mixed camphor with menthol
    \ex She introduced us $\Leftrightarrow$ She introduced you to me
    \ex She combined the beans and the rice $\Leftrightarrow$ She combined the beans with the rice
  \end{xlist}
\end{xlist}
\end{exe}
\citet{Winter:2018} hypothesizes that these predicates are also symmetric. A symmetric predicate is one in which the arguments, in a transitive, or the internal arguments, in a ditransitive, can be switched, preserving meaning.
\begin{exe}
\raggedright
  \ex \label{ex:symm} $bPc$ $\Leftrightarrow$ $cPb$ (symmetric predicates)
  \begin{xlist}
  \ex \label{ex:symmtrans} Transitives
  \begin{xlist}
    \ex Aisha dated Arjun $\Leftrightarrow$ Arjun dated Aisha
    \ex Your name rhymes with my name $\Leftrightarrow$ My name rhymes with your name
    \ex California is adjacent to Oregon $\Leftrightarrow$ Oregon is adjacent to California.
  \end{xlist}
  \ex Ditransitives
  \begin{xlist}
    \ex She mixed camphor with menthol $\Leftrightarrow$ She mixed menthol with camphor 
    \ex She introduced you to me $\Leftrightarrow$ She introduced me to you
    \ex She combined the beans with the rice $\Leftrightarrow$ She combined the rice with the beans
  \end{xlist}
\end{xlist}
\end{exe}
Let's call this Winter's Generalization.\footnote{In fact, \citet{Winter:2018} argues that (\ref{ex:WL}) should be a biconditional. But he notes that there are counterexamples for the implication [$b$P$c$ $\Leftrightarrow$ $c$P$b$] $\rightarrow$ [$a$P $\Leftrightarrow$ $b$P$c$], some of which are:
\begin{exe}
\raggedright
  \exi{(i)}
  \begin{xlist}
    \ex This resembles that. $\Leftrightarrow$ That resembles this. $\neq$ *They resemble.
    \ex She packaged this with that. $\Leftrightarrow$ She packaged that with this. $\neq$ She packaged them.
    \ex You concur with me. $\Leftrightarrow$ I concur with you $\neq$ We concur.
    \ex This compares with that. $\Leftrightarrow$ That compares with this. $\neq$ They compare.
    \ex You teamed with me. $\Leftrightarrow$ I teamed with you. $\neq$ We teamed.
  \end{xlist}
\end{exe}}
\begin{exe}
\raggedright
  \ex \label{ex:WL} Winter's Generalization\\
  Where $a$ = $b \oplus c$, [$a$P $\Leftrightarrow$ $b$P$c$] $\rightarrow$ [$b$P$c$ $\Leftrightarrow$ $c$P$b$]
\end{exe}
In (\ref{ex:external}), $a$ is a subject and $b$ and $c$ are external and internal arguments. In (\ref{ex:internal}), $a$ is a direct object, and $b$ and $c$ are internal arguments. We will propose a strategy for deriving Winter's Generalization.

Winter's own suggestion is to invoke a set of rules from proto-predicates---logical templates for verb meanings---to lexicalizations of those proto-predicates. He takes the $a$P form to be basic, and then provides rules for expressing P as a relation between individuals just in case certain conditions of similarity hold between the atoms of the plural $a$ and~P. His model requires that verbs denote relations between arguments and an event. The denotation for transitive \textit{date}, for instance, is~(\ref{ex:marrydenW}).
\begin{exe}
\raggedright
  \ex \label{ex:marrydenW} $\lambda x\ \lambda y\ \lambda \epsilon$ \textsc{date}($x,y,\epsilon$)
\end{exe}
It is built into his explanation that verbs can be relations between entities, and indeed, characterizing the property that verbs must have to participate in Winter's Generalization as ``symmetry'' also relies on taking verbs to be relations between entities. We believe the neo-Davidsonian project of expressing the meanings of verbs in terms of $\theta$-roles is successful, however, and on that view verbs never relate entities. Entities are related to events, and never each other. The neo-Davidsonian treatment of transitive \textit{date} is~(\ref{ex:marrydenD}).
\begin{exe}
\raggedright
  \ex \label{ex:marrydenD} $\lambda x\ \lambda y\ \lambda \epsilon$ \textsc{date}($\epsilon$) $\wedge$ \textsc{agent}($y$)($\epsilon$) $\wedge$ \textsc{agent}($x$)($\epsilon$)
\end{exe}
A neo-Davidsonian view of verb denotations doesn't offer a way of expressing a symmetric relation between arguments. 

Instead, we suggest that what makes symmetric predicates (as we'll continue to call them) special is that they allow two arguments to be related to the event by the same $\theta$-role. This is otherwise prevented: verbs do not come equipped with the ability to assign the same $\theta$-role to two different arguments. That verbs of this kind are generally forbidden is canonized in laws tailored to a variety of different frameworks. In \citet{Chomsky:1981}, it's called the Theta Criterion.
\begin{exe}
\raggedright
  \ex \label{ex:TC} Theta Criterion\\
  No $\theta$-role may be assigned to more than one argument position.
\end{exe}
We agree with \citet{Carlson:1984aa}, \citet{Dowty:1989}, \citet{Schein:1993}, and \citet{Williams:2015} that the Theta Criterion is a consequence of the meaning of predicates, and not a syntactic principle. That meanings are relevant is indicated by the fact that (\ref{ex:unique1}) doesn't entail~(\ref{ex:unique2}). (This argument comes from \citealt{Schein:1993}.)
\begin{exe}
\raggedright
  \ex \label{ex:unique}
  \begin{xlist}
    \ex \label{ex:unique1} Arjun and Aisha baked a cake.
    \ex \label{ex:unique2} Arjun baked a cake.
  \end{xlist}
\end{exe}
(\ref{ex:unique2}) describes an event that has Arjun as sole agent, and this isn't the event (\ref{ex:unique1}) describes. (\ref{ex:unique1}) describes scenarios in which either Arjun or Aisha are sole agents of a baking event, but also events in which they are only joint agents of a baking event. (Arjun sifts the flour, for instance, and Aisha cracks the eggs.) A DP that bears the agent $\theta$-role refers to all, and only, those entities that are agents of the event(s) being described. To derive this, we propose that the denotation of the \textsc{agent} $\theta$-role has the exhaustivity requirement defined in (\ref{ex:exhaust}) associated with it. The denotation of agent is therefore (\ref{ex:uniqueth1}), and the denotations for $\theta$-roles generally are as~(\ref{ex:uniqueth2}) describes. (The name ``role exhaustion'' comes from \citealt[chapter 8.2, p.~165]{Williams:2015}.)
\begin{exe}
\raggedright
  \ex \label{ex:uniqueth} Role Exhaustion
  \begin{xlist}
    \ex \label{ex:exhaust} \textsc{$\theta$}_{X}($a$)($\epsilon$) = \textsc{$\theta$}($a$)($\epsilon$) $\wedge$ $\forall y$ [\textsc{$\theta$}($y$)($\epsilon$) $\rightarrow\ y \leq a$] 
    \ex \label{ex:uniqueth1} \denotes{agent} = $\lambda x\ \lambda \epsilon$ \textsc{agent}_{X}($x$)($\epsilon$)
    \ex \label{ex:uniqueth2} \denotes{$\theta$} = $\lambda x\ \lambda \epsilon$ \textsc{$\theta$}_{X}($x$)($\epsilon$)
  \end{xlist}
\end{exe}
Role Exhaustion causes (\ref{ex:unique2}) to describe different events than (\ref{ex:unique1}) describes---it excludes those in which there is someone other than Arjun participating in the agent relation. This explains the failure of entailment between (\ref{ex:unique1}) and (\ref{ex:unique2}). It also takes a large step towards deriving the Theta Criterion, as  assigning the same $\theta$-role to two arguments would impose the restriction on each of them that they uniquely have that $\theta$-role.\footnote{Role Exhaustion would, unaided, allow violations of the Theta Criterion if the two arguments bearing the same $\theta$-role refer to the same entity.} We'll use Role Exhaustion in what follows to express the Theta Criterion, but we don't believe this is necessary for our argument here.

Symmetric predicates are ones that find a way of assigning the same $\theta$-role to two arguments without violating Role Exhaustion. We have~(\ref{ex:way1}).
\begin{exe}
\raggedright
  \ex \label{ex:way1} When a verb, P, allows $a$P $\Leftrightarrow$ $b$P$c$, it is because $a, b,$ and $c$ bear the same $\theta$-role. P is equipped with a way of overcoming Role Exhaustion.
\end{exe}
Winter's Generalization follows from positing that the arguments $P$ takes bear the same $\theta$-role, since~(\ref{ex:WRn}) is true.
\begin{exe}
\raggedright
  \ex \label{ex:WRn} If P assigns the same $\theta$-role to arguments $b$ and $c$, then $c$P$b$ $\Leftrightarrow$ $b$P$c$.
\end{exe}

The remainder of this paper will focus on delivering the way by which verbs can overcome Role Exhaustion. A useful place to look is in certain kinds of comitative constructions, like~(\ref{ex:com1}).
\begin{exe}
\raggedright
  \ex \label{ex:com1} Aisha ran with Arjun.
  \begin{xlist}
    \ex $\approx$ Aisha and Arjun are both agents of the same running event.
    \ex $\approx$ Aisha is the agent of the running event and Arjun accompanies Aisha.
  \end{xlist}
\end{exe}
We believe that (\ref{ex:com1}) is ambiguous---given by the paraphrases---and only one of its interpretations is relevant to our project. The irrelevant interpretation is (\ref{ex:com1}b), which arises when (\ref{ex:com1}) is used to report that Aisha ran with her son, Arjun, strapped to her back. We suspect that this meaning arises because the \textit{with}-phrase can be treated as a depictive, like~(\ref{ex:dep1}).
\begin{exe}
\raggedright
  \ex \label{ex:dep1}
  Aisha ran mindful of Arjun.
\end{exe}
It is possible to remove this interpretation of the \textit{with}-phrase by adding \textit{together}, whose semantics requires that the predicate it combines with have a plural argument. 
\begin{exe}
\raggedright
  \ex \label{ex:togetherbase}
  \begin{xlist}
    \ex[]{They ran together.}
    \ex[*]{She ran together.}
  \end{xlist}
\end{exe}
Thus:
\begin{exe}
\raggedright
  \ex \label{ex:together1}
  \begin{xlist}
    \ex[*]{Aisha ran together mindful of Arjun.}
    \ex[]{Aisha ran together with Arjun.}
    \begin{xlist}
      \ex $\approx$ Aisha and Arjun are agents of the running event
      \ex $\neq$ Aisha is the agent of the running event and Arjun accompanies Aisha.
    \end{xlist}
  \end{xlist}
\end{exe}

We learn two things from these comitative examples.
\begin{exe}
\raggedright
  \ex \label{ex:2things}
  \begin{xlist}
    \ex A comitative can have a plural argument even when there is only a singular on the surface.
    \ex It is possible to create a symmetric predicate from a non-symmetric predicate with this plural comitative.
    \begin{xlist}
      \sn \textit{witness:}\\ Aisha ran together with Arjun $\Leftrightarrow$ Arjun ran together with Aisha
    \end{xlist}
  \end{xlist}
\end{exe}
Proposal:
\begin{exe}
\raggedright
  \ex \label{ex:withprop} \textit{with} can take two arguments, form a plural from them, and relate that plural to an event.
\end{exe}
Let's develop (\ref{ex:withprop}).

We start by considering how a \textit{with} phrase can be the predicate of a copular construction.
\begin{exe}
\raggedright
  \ex \label{ex:withcop} Aisha is with Arjun.
  \ex \label{ex:withden} \denotes{$\surd$with} = \parbox[t]{3in}{$\lambda P\ \lambda x\ \lambda y\ \lambda \epsilon$  $P$($x \oplus y$)($\epsilon$)}
  \ex \label{ex:withcopt}
  \small
  \begin{forest}
  [$\lambda \epsilon$ $R$(Arjun $\oplus$ Aisha)($\epsilon$)\\TP
    [DP^2 [Aisha, roof]]
    [$\lambda 2\ \lambda \epsilon$ $R$(Arjun $\oplus$ 2)($\epsilon$) \\TP
      [T [is]]
      [$\lambda \epsilon$ $R$(Arjun $\oplus$ 2)($\epsilon$) \\PP
        [\textit{t_2} ]
        [$\lambda y\ \lambda \epsilon$ $R$(Arjun $\oplus y$)($\epsilon$) \\PP
          [P\\ (with) [$\lambda P\ \lambda x\ \lambda y\ \lambda \epsilon$  $P$($x \oplus y$)($\epsilon$)\\ $\surd$with] [$R$]]
          [DP [Arjun, roof]]
         ]
       ]
     ]
  ]
  \end{forest}
\end{exe}
As (\ref{ex:withcopt}) indicates, the denotation of (\ref{ex:withcop}) is a predicate that describes eventualities which the sum of Aisha and Arjun have the relation, $R$, to. $R$ is a free variable, whose value is influenced by context. In (\ref{ex:R}) are some illustrative contexts, and the accompanying value for $R$ that they favor.
\begin{exe}
\raggedright
  \ex \label{ex:R}
  \begin{xlist}
    \ex
    \begin{xlist}
    \exi{Q:} Where is Aisha?
    \exi{A:} She is with Arjun.\\ $R$ $\approx$ $\lambda x\ \lambda \epsilon$ \textsc{at}($x$)($\epsilon$) $\wedge$ \textsc{location}($\epsilon$)
    \end{xlist}
    \ex
    \begin{xlist}
      \exi{Q:} Who will dance together at the contest?
      \exi{A:} Aisha is with Arjun and I am with you.\\ $R$ $\approx$ $\lambda x\ \lambda \epsilon$ \textsc{participant}($x$)($\epsilon$) $\wedge$ \textsc{dance}($\epsilon$)
    \end{xlist}
    \ex 
    \begin{xlist}
      \exi{Q:} Is anyone else voting for the measure?
      \exi{A:} I'm with you!\\ $R$ $\approx$ $\lambda x\ \lambda \epsilon$ \textsc{agent}($x$)($\epsilon$) $\wedge$ \textsc{vote}($\epsilon$)
    \end{xlist}
    \ex 
    \begin{xlist}
      \exi{Q:} Is Aisha single?
      \exi{A:} She is with Arjun.\\ $R$ $\approx$ $\lambda x\ \lambda \epsilon$ \textsc{participant}($x$)($\epsilon$) $\wedge$ \textsc{relationship}($\epsilon$)
    \end{xlist}
  \end{xlist}
\end{exe}

We have put $R$ into the syntax, so that it can be moved (or bound), for this is how we suggest \textit{with}-phrases compose with VPs when they form symmetric predicates. (\ref{ex:rancon1}) illustrates how \textit{with} composes with an intransitive verb, and (\ref{ex:rancon2}) illustrates how \textit{with} composes with a transitive.
\begin{exe}
\raggedright
  \ex \label{ex:rancon1} \ \\
  \bigskip

  \footnotesize
  \hspace*{-50pt}
  \begin{forest}
  [$\lambda \epsilon$ \textsc{agent}_{X}(Arjun $\oplus$ Aisha)($\epsilon$) $\wedge$ \textsc{run}($\epsilon$)  \\TP
    [DP^2 [Aisha, roof]]
    [TP
      [T [past]]
      [$\lambda \epsilon$ \textsc{agent}_{X}(Arjun $\oplus\ 2$)($\epsilon$) $\wedge$ \textsc{run}($\epsilon$)  \\vP
        [$\lambda x\ \lambda \epsilon$ \textsc{agent}_{X}($x$)($\epsilon$) $\wedge$ \textsc{run}($\epsilon$) \\vP
          [v [agent]]
          [VP [V [$\surd$run]]]
         ]
        [$\lambda R\ \lambda \epsilon$  $R$(Arjun $\oplus\ 2$)($\epsilon$)\\PP
          [$\lambda R$ ]
          [$\lambda \epsilon$  $R$(Arjun $\oplus\ 2$)($\epsilon$)\\PP
            [\textit{t_2} ]
            [$\lambda y\ \lambda \epsilon$  $R$(Arjun $\oplus\ y$)($\epsilon$)\\PP
              [$\lambda x\ \lambda y\ \lambda \epsilon$  $R$($x \oplus\ y$)($\epsilon$)\\ P\\ (with)
                [$\lambda P\ \lambda x\ \lambda y\ \lambda \epsilon$  $P$($x \oplus\ y$)($\epsilon$)\\ $\surd$with ]
                [$R$ ]
               ]
              [DP [Arjun, roof]]
             ]
           ]
         ]
       ]
     ]
  ]
  \end{forest}
  \normalsize \sn = \textit{Aisha ran with Arjun.}
  \ex \label{ex:rancon2}\ \\
  \bigskip

  \footnotesize
  \hspace*{-160pt}
  \begin{forest}
  [$\lambda \epsilon$ \textsc{agent}_{X}(she)($\epsilon$) $\wedge$ \textsc{theme}_{X}(menthol $\oplus$ camphor)($\epsilon$) $\wedge$ \textsc{mix}($\epsilon$) \\TP
    [DP^2 [she, roof]]
    [TP
      [T [past]]
      [$\lambda \epsilon$ \textsc{agent}_{X}($2$)($\epsilon$) $\wedge$ \textsc{theme}_{X}(menthol $\oplus$ camphor)($\epsilon$) $\wedge$ \textsc{mix}($\epsilon$) \\vP
        [\textit{t_2} ]
        [$\lambda x\ \lambda \epsilon$ \textsc{agent}_{X}($x$)($\epsilon$) $\wedge$ \textsc{theme}_{X}(menthol $\oplus$ camphor)($\epsilon$) $\wedge$ \textsc{mix}($\epsilon$) \\vP
          [v [agent]]
          [$\lambda \epsilon$ \textsc{theme}_{X}(menthol $\oplus$ camphor)($\epsilon$) $\wedge$ \textsc{mix}($\epsilon$)\\VP
            [$\lambda x\ \lambda \epsilon$ \textsc{theme}_{X}($x$)($\epsilon$) $\wedge$ \textsc{mix}($\epsilon$)\\V [$\surd$mix]]
            [PP
              [$\lambda R$]
              [$\lambda \epsilon$ $R$(menthol $\oplus$ camphor)($\epsilon$) \\PP
                [DP [camphor, roof]]
                [$\lambda y\ \lambda \epsilon$ $R$(menthol $\oplus\ y$)($\epsilon$) \\PP
                  [$\lambda x\ \lambda y\ \lambda \epsilon$ $R$($x \oplus\ y$)($\epsilon$) \\P\\ (with)
                    [$\lambda P\ \lambda x\ \lambda y\ \lambda \epsilon\ P(x \oplus\ y)$($\epsilon$)\\ $\surd$with]
                    [$R$ ]
                   ]
                  [DP [menthol, roof]]
                 ]
               ]
             ]
           ]
         ]
       ]
     ]
  ]
  \end{forest}
  \normalsize \sn \textit{= She mixed camphor with menthol.}
\end{exe}
Notice that this treatment of \textit{with}-phrases gives them a plural argument, and thereby makes them eligible to be modified by \textit{together}.

We speculate that symmetric predicates that don't involve the word \textit{with} are built from a syntax that nonetheless contains \denotes{$\surd$with}. That syntax puts the denotation of \textit{$\surd$with} inside the word that is pronounced as the verb. (\ref{ex:symmfintrans}) and (\ref{ex:symmfinditrans}) illustrate. We give both frames for these verbs, to illustrate how the $aP$ $\Leftrightarrow$ $bPc$ alternation is modeled.
\pagebreak
\begin{exe}
\raggedright
  \ex \label{ex:symmfintrans} $bPc$ frame:\\[8pt]
  % \small
  \begin{forest}
      [$\lambda \epsilon$ \textsc{agent}_{X}(Arjun $\oplus$ Aisha)($\epsilon$) $\wedge$ \textsc{date}($\epsilon$) \\vP
        [DP [Aisha, roof]]
        [$\lambda y\ \lambda \epsilon$ \textsc{agent}_{X}(Arjun $\oplus\ y$)($\epsilon$) $\wedge$ \textsc{date}($\epsilon$) \\vP
          [$\lambda x\ \lambda y\ \lambda \epsilon$ \textsc{agent}_{X}$(x \oplus\ y)(\epsilon)$ $\wedge$ \textsc{date}($\epsilon$) \\v\\ (\textit{date})
          [$\lambda z\ \lambda \epsilon$ \textsc{agent}_{X}($z$)($\epsilon$) $\wedge$ \textsc{date}($\epsilon$)\\v
            [v [agent]]
            [V [$\surd$date]]
          ]
          [$\lambda P\ \lambda x\ \lambda y\ \lambda \epsilon$ $P(x \oplus\ y)(\epsilon)$ \\$\surd$with]
          ]
          [DP [Arjun, roof]]
         ]
       ]
  \end{forest}\\
  \normalsize \sn \textit{=Aisha dated Arjun.}
  \medskip

  \normalsize \ex $aP$ frame:\\[8pt]
  % \small
  \begin{forest}
  [$\lambda \epsilon$ \textsc{agent}_{X}(Aisha $\oplus$ Arjun)($\epsilon$) $\wedge$ \textsc{date}($\epsilon$) \\vP
    [DP [Aisha and Arjun, roof]]
    [$\lambda x\ \lambda \epsilon$ \textsc{agent}_{X}($x$)($\epsilon$) $\wedge$ \textsc{date}($\epsilon$) \\vP
      [v [agent]]
      [VP [V [$\surd$date]]]
     ]
  ]
  \end{forest}\\
    \normalsize \sn \textit{= Aisha and Arjun dated}
\pagebreak
   \ex \label{ex:symmfinditrans} $bPc$ frame:\\[8pt]
  % \small
  % \hspace*{-40pt}
\begin{forest}
    [$\lambda \epsilon$ \textsc{agent}_{X}(she)($\epsilon$) $\wedge$ \textsc{theme}_{X}(you $\oplus$ me)($\epsilon$) $\wedge$ \textsc{introduce}($\epsilon$) \\vP
      [DP [she, roof]]
      [$\lambda x\ \lambda \epsilon$ \textsc{agent}_{X}($x$)($\epsilon$) $\wedge$ \textsc{theme}_{X}(you $\oplus$ me)($\epsilon$) $\wedge$ \textsc{introduce}($\epsilon$) \\vP
        [v [agent]]
        [$\lambda \epsilon$ \textsc{theme}_{X}(you $\oplus$ me)($\epsilon$) $\wedge$ \textsc{introduce}($\epsilon$)\\VP
          [$\lambda y\ \lambda \epsilon$ \textsc{theme}_{X}(you $\oplus\ y)$($\epsilon$) $\wedge$ \textsc{introduce}($\epsilon$)\\VP
          [$\lambda x\ \lambda y\ \lambda \epsilon$ \textsc{theme}_{X}$(x \oplus\ y)$($\epsilon$) $\wedge$ \textsc{introduce}($\epsilon$)\\V\\ (introduce)
            [$\lambda z\ \lambda \epsilon$ \textsc{theme}_{X}($z$)($\epsilon$) $\wedge$ \textsc{introduce}($\epsilon$)\\V [$\surd$introduce]]
            [$\lambda P\ \lambda x\ \lambda y\ \lambda \epsilon$ $P(x \oplus\ y)$($\epsilon$)\\ $\surd$with ]
           ]
          [DP [you, roof]]
         ]
         [PP [to me, roof]]
       ]
     ]
   ]
\end{forest}\\
\sn \textit{= She introduced you to me.}
\medskip

    \ex $aP$ frame:\\[8pt]
    % \small
    \begin{forest}
    [$\lambda \epsilon$ \textsc{agent}_{X}(she)($\epsilon$) $\wedge$ \textsc{theme}_{X}(us)($\epsilon$) $\wedge$ \textsc{introduce}($\epsilon$) \\vP
      [DP [she, roof]]
      [$\lambda x\ \lambda \epsilon$ \textsc{agent}_{X}($x$)($\epsilon$) $\wedge$ \textsc{theme}_{X}(us)($\epsilon$) $\wedge$ \textsc{introduce}($\epsilon$) \\vP
        [v [agent]]
        [$\lambda \epsilon$ \textsc{theme}_{X}(us)($\epsilon$) $\wedge$ \textsc{introduce}($\epsilon$) \\VP
          [$\lambda x\ \lambda \epsilon$ \textsc{theme}_{X}($x$)($\epsilon$) $\wedge$ \textsc{introduce}($\epsilon$) \\V [$\surd$introduce]]
          [DP [us, roof]]
         ]
       ]
    ]
    \end{forest}\\
    \normalsize \textit{= She introduced us.}
\end{exe}
Note that we take \textit{to} to be semantically vacuous.

Our treatment of \textit{with}-less symmetric predicates is similar to a popular account of the causative/inchoative alternation. On that account, the alternation in (\ref{ex:causincho1}) is syntactically parallel to the contrast in~(\ref{ex:causincho2}).
\begin{exe}
\raggedright
  \ex \label{ex:causincho1}
  \begin{xlist}
    \ex Aisha bounced the ball.
    \ex The ball bounced.
  \end{xlist}
  \ex \label{ex:causincho2}
  \begin{xlist}
    \ex Aisha made the ball bounce.
    \ex The ball bounced.
  \end{xlist}
\end{exe}
The terms associated with  \denotes{make} in (\ref{ex:causincho2}a) are syntactically present in (\ref{ex:causincho1}a), but are exponed by the verb \textit{bounce}. Similarly, we suggest that the denotation for \textit{$\surd$with} is present in examples like (\ref{ex:symmfintrans}) and (\ref{ex:symmfinditrans}), but exponed by the verbs \textit{date} and \textit{introduce}. The verbs that participate in the $aP$ $\Leftrightarrow$ $bPc$ alternation are just those predicates that English has decided can expone~\textit{$\surd$with}. When they expone \textit{$\surd$with}, they have overcome Role Exhaustion, and are therefore symmetric. This is Winter's Generalization.




%References
\printbibliography[title={\sffamily References}]

\end{document}

``when reciprocal predicates show this equivalence, we refer to them as plain reciprocals'':
(4) Sue and Dan dated <==> Sue dated Dan (Dan dated Sue)
p. 3

P1. Meanings of symmetric predicates are logically derived from the collective meanings of their reciprocal alternates. (p.)

"To account for this pattern, it is proposed that the basic lexical meaning of symmetric binary predicates in English is unary-collective." (p. 4)

``symmetry is not an accidental semantic feature of binary predicates, but appears by virtue of their underlying collectivity.'' (p. 4)

Plain Reciprocity
for all x,y in E, such that x=\=y:P(X+y) <==> R(x,y) and R(y,x) (p. 5, (7))

 Reciprocity-symmetry Generalization: A reciprocal alternation between a unary collective predicate P and a binary R is \uline{plain} if and only if R is truth-conditionally symmetric. ((18), p. 11)

  ``...we logically derive the meaning of the binary construction from the collective meaning.'' (p. 14)

p. 24:
  A and B are in love ==> A is in love with B and B is in love with A, but A is in love with B and B is in love with A =\=> A&B are in love. ("talking to" is similar?)
  -- thought: could this be what happens with stative predicates that are like "hug"?

p. 27
Winter suggests that all these ``hug'' like examples obey this rule: P(x+y) ==> R(x,y) or R(y,x).

 p. 29: ``...transitive verbs without reciprocal alternates individuate single events.'' Sue pushed Dan and Dan pushed Sue ==> two pushes.

 